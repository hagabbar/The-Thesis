\chapter{Abstract}

Over 100 years ago Einstein formulated his now famous theory of 
General Relativity. In his theory he discovered a wave-like solution 
to the field equations which lead to the beginning of a brand-new 
astronomical field, \ac{GW} astronomy. 

The Laser Interferometer Gravitational Wave Observatory (LIGO), Virgo, 
KAGRA (LVK) Collaboration's main focus is the detection of gravitational 
wave events from some of the most violent and cataclysmic events 
in the known universe. The LVK detectors are composed of 
large-scale Michelson Morley interferometers able to detect 
strain amplitudes on the order of $h \sim 10^{-21}$ from a range 
of sources including binary black holes, binary neutron stars, 
neutron star black holes, supernovae and stochastic gravitational 
waves. Although these events release an incredible amount of 
energy, the amplitudes of the \ac{GW}s from such events 
is also incredibly small. 

The LVK uses sophisticated techniques such as matched 
template filtering and Bayesian inference in order 
to both detect and infer source parameters 
from \ac{GW} events. Although optimal under most 
cirucumstances, these standard methods are computationally 
expensive to use. A solution to reducing the computational 
expense of such techniques is to use machine learning.

In this thesis, I will show how we have employed 
various machine learning and statistical techniques 
in order to both optimize the search and show 
that such machine learning techniques are indeed just 
as efficient as the gold standard approaches currently 
used in the collaboration. The first two chapters 
will give a brief introduction to general relativity 
as it relates to gravitational wave astronomy, \ac{GW}s, 
standard \ac{GW} data analysis methods and machine 
learning. Chapters 


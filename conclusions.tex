\chapter{Conclusions and Future Work}\label{ch:chap_6}

In this thesis it was shown for the first time how machine learning may 
be used for both \ac{GW} signal detection and parameter estimation. 
Whilst current state-of-the-art algorithms used by the \ac{LVK} are 
optimal to a certain degree for both 
the detection and parameter estimation of \acp{GW} (matched filtering 
and Bayesian inference respectively), the number of signals we expect to 
see in future observation runs will dwarf the computational 
resources needed for continued use of such techniques. The 
work in this thesis has shown several first of its kind proof-of-principal 
studies which illustrate that \ac{ML} techniques are a powerful and accurate tool for signal detection and parameter estimation.

%
% Summarize each chapter
%

% Chapter 1 and 2
In Ch.~\ref{ch:chap_1} I introduced the basic concepts of 
\ac{GR} and how \acp{GW} arise from said concepts.  
I discussed standard techniques for \ac{GW} detection and parameter 
estimation (matched template filtering and Bayesian inference respectively). 
I showed how the \ac{LIGO}/Virgo detectors detect \acp{GW}, as well as account 
for 
various noise sources during operation. We also discussed that while matched 
filtering and Bayesian parameter estimation are generally efficient and 
reliable, they can also be computationally expensive. We then provided an 
overview of \ac{GW} detections made by the \ac{LVK} up to the present day and 
how future observation runs will produce an abundance of \ac{GW} signals. 
Given the expected number of signals in future observing runs, 
it was made clear in this 
introductory chapter that there is a need for efficient low-latency signal 
detection and parameter estimation methods.  

% Chapter 2
In Ch.~\ref{ch:chap_2} we start by introducing the simple concept of 
a perceptron neural network. From there, I showed how the concept of a 
perceptron may be expanded to include multiple perceptrons in a layer and multiple 
layers of perceptrons into a deep neural network. It was also discussed how 
a deep neural network is trained and how one may then construct other 
neural network types, such as \acp{CNN}, \acp{AE}, \acp{VAE} and \acp{CVAE}. 
With the broad range of introductory \ac{ML} material covered in this chapter, 
I was then able to discuss how many of these concepts have been (and continue 
to be) applied several domains within the field of \ac{GW} astronomy in the 
following chapter (Ch.~\ref{ch:chap_3}).

% Chapter 3
In Ch.~\ref{ch:chap_3} I performed a survey of various \ac{ML} 
techniques employed in \ac{GW} astronomy. It was seen that \ac{ML} has 
had a surge in use in only the recent past and has been applied 
to a variety of tasks including glitch classification, 
\ac{GW} population analysis, \ac{GW} Bayesian parameter 
estimation and point estimate parameter estimation, 
signal detection and \ac{PSD} estimation. Given the wide scope of tasks 
in which neural networks have been applied across the whole of \ac{GW} 
astronomy, it is perhaps an indication that \ac{ML} will play an 
increasingly pivotal role in the \ac{LVK} over the coming years.

% Chapter 4
In Ch.~\ref{ch:chap_4} I presented my study in which I used \acp{CNN} to 
perform \ac{GW} signal detection. Given that the expected number 
of detections made by the \ac{LVK} is expected to increase dramatically 
in the coming years, there is an urgent need for faster signal detection 
techniques. A \ac{CNN}, is one such technique which is able to 
produce predictions on signal classes in low-latency. In my study I trained
a \ac{CNN} model on 
\ac{BBH} signals buried in whitened Gaussian noise. The 
results from the \ac{CNN} model were compared to those from the standard 
signal detection method used in the \ac{LVC} (matched filtering). 
Results between both the \ac{ML} approach and the matched filtering
were seen to be in strong agreement with each other. These results 
showed for the first time that deep learning could match the efficiency 
of exisiting signal detection 
techniques and as such contributed to an increased level 
of confidence within the \ac{GW} astronomy community 
that deep learning could be a powerful tool. 

% Chapter 5
In Ch.~\ref{ch:chap_5} I presented my study in which I used \acp{CVAE} to 
produce Bayesian posteriors on 15 \ac{BBH} parameters. Our \ac{CVAE} 
model (\texttt{VItamin}) was trained on whitened \ac{BBH} signals 
injected in Gaussian noise. We also produced benchmark results 
using known and trusted Bayesian nested sampling and \ac{MCMC} 
samplers (\texttt{Dynesty}, \texttt{Ptemcee}, \texttt{CPNest}, 
\texttt{Emcee}). Results from both \texttt{VItamin} and Bayesian samplers 
were compared against each other using many different figures of merit such 
as the \ac{JS}-divergence between posteriors and \ac{PP} plots. It was 
shown that results from the \ac{ML} approach had similar levels 
of agreement as those between Bayesian samplers against other Bayesian 
samplers. It was also shown that the speed of \texttt{VItamin} 
was $\sim 6$ orders of magnitude greater than those of the Bayesian 
samplers, producing posteriors in less than 1~s. Given the speed gains 
from this approach and the accuracy of \acp{CVAE} with respect traditional 
samplers, it is evident that \acp{CVAE} may be useful for low-latency
\ac{EM} follow-up analyses. The low-latency data products from \acp{CVAE} 
will allow for the observation of prompt \ac{EM} emission on timescales 
which are far shorter than exisiting Bayesian sampling techniques. 
It was also stated in this chapter that the \ac{ML} model used may 
be extended to a variety of other source types and may also in future work 
be trained/tested using real non-Gaussian detector noise.

%
% Future work
%
\ac{ML} has been used in countless applications across a wide 
variety of fields including: 
economics~\cite{Alexakis2021,Babaei2021,Bouri2021},
biology~\cite{10.3389/fgene.2019.00214,Xu2019,10.1371/journal.pcbi.0030116}
, chemistry~\cite{Tkatchenko2020,Deringer2020,Margraf2021} 
and physics~\cite{Manzhos_2020,2002.09405,50295}. \ac{GW} astronomy has 
only in the past several years seen an increase in the number of studies 
which apply \ac{ML} to a wide variety of applications including: \ac{GW} 
signal detection, \ac{GW} parameter estimation, and many other 
studies (See Ch.~\ref{ch:chap_3}). I have shown in this thesis several first proof-of-principal studies which show that \ac{ML} can match the accuracy of standard trusted techniques currently used in the collaboration for both signal detection and parameter estimation. Going forward, there are many future directions we could take with this work. 

%
% signal detection future work
%
At the time of publication for our signal detection
paper~\cite{PhysRevLett.120.141103}, we mentioned that one could 
apply our method (or similar \ac{ML} models) to sources 
other than \ac{BBH} signals including: \ac{BNS}, \ac{NSBH}, 
\ac{CW} and stochastic \ac{GW} events. In the subsequent 
years, there have been many follow-up studies which have 
demonstrated success using a variety of \ac{ML} algorithms for 
\ac{BNS} signal detection \cite{PhysRevD.102.063015,
KRASTEV2020135330,Lin2019}, 
\ac{BNS} early-warning signal detection \cite{WEI2021136185}, 
\ac{CW} detection \cite{PhysRevD.100.044009, PhysRevD.102.083024}, 
and stochastic background detection \cite{9461904}. Moreover, 
it has been shown that unmodelled \ac{ML} approaches may also 
be used to detect bust-like signals~\cite{Cavagli__2020,PhysRevD.102.043022}.
Given the success of many of the aforementioned studies, there is 
still much work left to be done in the regime of testing such 
algorithms on real data in real noise (including our own work). 
There is also much interest in the community with regards to \acp{FAR} 
of such algorithms and there should be more studies done concerning what 
classes of spurious non-astrophysical signals may ``trick'' 
\ac{ML} models into outputting false detections. I also mentioned the possibilities of using \acp{CNN} for point-estimate parameter estimation 
in my signal detection paper~\cite{PhysRevLett.120.141103}, 
but acknowledge that this may not be the most useful when 
compared with recent developments in \ac{ML} for Bayesian parameter 
estimation which returns the posterior directly.

There have also been many recent developments in the field of 
\ac{ML} hyperparameter
optimisation~\cite{li2019generalized,1206.2944,NIPS2011_86e8f7ab}
(Ch.~\ref{ch:chap_2}, Sec.~\ref{sec:hyperpar_optim}), and such 
optimisation algorithms have become more widely available and 
easier to use~\cite{omalley2019kerastuner}. It would be interesting 
to see whether the use of such algorithms can outperform a  
random hyperparameter search used by most \ac{GW} astronomy 
\ac{ML} practitioners today. 
It should also be noted that the gap must be 
bridged between proof-of-principal studies to the development of 
easy-to-use packaged \ac{ML} \ac{GW} detection pipelines in order to 
make the most use of these approaches. 

% Parameter estimation future work
Following the initial release of my study on Bayesian parameter 
estimation using \acp{CVAE}, there has been a flurry of follow-up 
studies which have used many alternative \ac{ML} techniques to 
produce similar data products. Such work includes those done
by~\cite{2019arXiv190905966C} using reduced order modeling and 
perceptron networks and also~\cite{PhysRevD.102.104057} using normalising 
flows. There have also been studies which use hybrid normalising flow - 
nested sampling algorithms in order to boost sampler 
convergence times~\cite{PhysRevD.103.103006}, as well as studies using \acp{CVAE} 
which have built upon my own work~\cite{2021arXiv210710730K}. We also note 
that there has been exciting recent developments which have shown for the 
first time that \ac{ML} models (normalising flows) may be used to 
accurately estimate Bayesian posteriors for real \ac{GW} signals in 
real noise~\cite{2008.03312}. Applying variational 
inference algorithms such as \acp{CVAE} and normalising flows to 
\ac{BNS} signals should also be explored, given the expected 
increase in detected \ac{GW} signals by the \ac{LVK}. Some issues that will 
need to be overcome will be the large length of \ac{BNS} signals 
($\sim100$s or more depending on the lower cut-off frequency) and 
%cut-off in sensitivity - for a BNS with 25Hz the length is 90s, 
%10Hz cut-off the length is 1000s, for 5Hz it's around 2 hours, 
%and for ET at 1 Hz it would be about 5 days.}). 
the large standard sampling rate of preprocessed data used by data analysis 
pipelines ($\sim8$kHz), where the computational power needed to train 
may be large. Possible solutions could be to use similar techniques as 
those employed by~\cite{PhysRevD.102.104057} like principal component analysis 
to reduce the dimensionality of the \ac{GW} timeseries, or those 
employed by~\cite{PhysRevD.102.063015} who used varying sample rates at 
different intervals of the input signal. As mentioned earlier in 
Ch.~\ref{ch:chap_4} Sec.~\ref{sec:vit_data_aug}, we also apply 
data augmentation to phase, distance and 
time of coalescence. It should be noted that this form of signal augmentation 
can be extended in future work to include more parameters and 
may therefore reduce the need for a large training set.

%
% Changing PSD discussions
%
It is known that the \ac{PSD} of \ac{GW} detectors varies as a function of 
time over the course of a given observation run. A changing noise floor 
can have adverse impact on the performance of neural networks 
which have been pre-trained on a \ac{PSD} which does not encompass 
such variance. One solution may be to provide many different 
\ac{PSD} realisations within the bounds of what the variance of 
the \ac{PSD} is to be expected over the course of an observing run as 
input to the neural network during training, thus marginalizing 
over the \ac{PSD}. Careful decisions would then have to be 
made as to how often one should re-train such a model as an 
observing run progresses. We also acknowledge the option 
to use the estimated \ac{PSD} for each training/validation/testing 
sample as a conditional input together with the data, as was done 
by~\cite{2021arXiv210612594D} using normalising flows. 

Re-training periodically may also 
benefit from the use of transfer learning~\cite{Goodfellow-et-al-2016} 
whereby previous iterations of the neural network model are used 
as the initialised values for the new neural network's weights. It 
is also important to produce packaged, reviewed pipelines of 
variational inference \ac{GW} algorithms, so that their benefits 
may be more fully utilised by others in the \ac{GW} astronomical 
community. Extending the \texttt{VItamin} analysis we 
presented in Ch.~\ref{ch:chap_5}, to larger prior ranges, as well as 
increasing the sample rate will be necessary in order to confirm 
the degree to which the computational cost may or may not increase 
for more complicated signals.

%
% Discussion on non-Gaussianity
%
It is also acknowledged that one of the 
limiting factors of my own work has been the use of Gaussian noise, rather 
than realistic non-Gaussian noise. If the \ac{ML} models used by myself for 
signal detection and parameter estimation were trained on real data, 
then (in principle) the \ac{ML} model would be able to learn the true likelihood 
and not be limited to the analytically imposed version. Assuming this to 
be the case, it may be possible that \ac{ML} will be able to 
exceed the performance of existing methods in non-Gaussian noise 
both in terms of speed and accuracy \hunter{citation would be nice}.
It is my hope that the next generation 
of Ph.D. students and/or postdocs will apply \acp{CVAE} under realistic 
noise conditions and test on real \ac{GW} signals in  
future work.

% Final thoughts
It is my hope that my own work will continue to be expanded upon 
and serve as a foundation for future studies which aim to provide rapid \ac{ML} 
data products, such as signal detection statistics and Bayesian posteriors
, in real time to scientists both inside and outside of the \ac{LVK}. Given 
that \ac{ML} studies for \ac{GW} astronomy have largely focused 
on \ac{CBC} events, possibly due in-part to the fact that 
they are the only sources 
we have detected so far, I believe that \ac{ML} will also have a crucial role to 
play regarding sources we have yet to detect or remain unmoddeled. It is my 
belief that rapid data products from \ac{ML} will only 
enhance current and future analyses, hopefully 
continuing to advance the field of \ac{GW} data analysis.